%%%%%%%%%%%%%%%%%%%%%%%%%%%%%%%%%%%%%%%%%
% Programming/Coding Assignment
% LaTeX Template
%
% This template has been downloaded from:
% http://www.latextemplates.com
%
% Original author:
% Ted Pavlic (http://www.tedpavlic.com)
%
% Note:
% The \lipsum[#] commands throughout this template generate dummy text
% to fill the template out. These commands should all be removed when
% writing assignment content.
%
% This template uses a Perl script as an example snippet of code, most other
% languages are also usable. Configure them in the "CODE INCLUSION
% CONFIGURATION" section.
%
%%%%%%%%%%%%%%%%%%%%%%%%%%%%%%%%%%%%%%%%%

%----------------------------------------------------------------------------------------
% PACKAGES AND OTHER DOCUMENT CONFIGURATIONS
%----------------------------------------------------------------------------------------

\documentclass{scrartcl}

\usepackage[T1]{fontenc}
\usepackage[utf8x]{inputenc}

\usepackage{fancyhdr} % Required for custom headers
\usepackage{lastpage} % Required to determine the last page for the footer
\usepackage{extramarks} % Required for headers and footers
\usepackage[usenames,dvipsnames]{color} % Required for custom colors
\usepackage{graphicx} % Required to insert images
\usepackage{listings} % Required for insertion of code
\usepackage{courier} % Required for the courier font

\usepackage{hyperref}

\linespread{1.1} % Line spacing

% Set up the header and footer
\pagestyle{fancy}
\lhead{\hmwkAuthorName \\} % Top left header
\chead{\hmwkClass: \hmwkTitle} % Top center head
\rhead{\firstxmark} % Top right header
\lfoot{\lastxmark} % Bottom left footer
\cfoot{} % Bottom center footer
\rfoot{Page\ \thepage\ of\ \protect\pageref{LastPage}} % Bottom right footer
\renewcommand\headrulewidth{0.4pt} % Size of the header rule
\renewcommand\footrulewidth{0.4pt} % Size of the footer rule

\setlength\parindent{0pt} % Removes all indentation from paragraphs

%----------------------------------------------------------------------------------------
%	CODE LISTINGS SETUP
%----------------------------------------------------------------------------------------
\usepackage{color}
\usepackage[usenames,dvipsnames,svgnames,table]{xcolor}
\colorlet{MAROON}{Maroon}
\usepackage{minted}
\providecommand*{\listingautorefname}{Listing}

%----------------------------------------------------------------------------------------
% NAME AND CLASS SECTION
%----------------------------------------------------------------------------------------

\newcommand{\hmwkTitle}{Project 1 -- Using Minimax with Alpha-Beta pruning to play Quarto} % Assignment title
\newcommand{\hmwkDueDate}{Wednesday,\ September\ 18,\ 2013} % Due date
\newcommand{\hmwkClass}{IT3105} % Course/class
\newcommand{\hmwkClassTime}{} % Class/lecture time
\newcommand{\hmwkClassInstructor}{Lecturer: Keith Downing} % Teacher/lecturer
\newcommand{\hmwkAuthorName}{Pablo Liste Garcia \& Dominik Horb} % Your name

%----------------------------------------------------------------------------------------
% TITLE PAGE
%----------------------------------------------------------------------------------------

\title{
\vspace{2in}
\textmd{\textbf{\hmwkClass:\ \hmwkTitle}}\\
\normalsize\vspace{0.1in}\small{Due\ on\ \hmwkDueDate}\\
\vspace{0.1in}\large{\textit{\hmwkClassInstructor\ \hmwkClassTime}}
\vspace{3in}
}

\author{\textbf{\hmwkAuthorName}}
\date{} % Insert date here if you want it to appear below your name

%----------------------------------------------------------------------------------------

\begin{document}

\maketitle

%----------------------------------------------------------------------------------------
% TABLE OF CONTENTS
%----------------------------------------------------------------------------------------

%\setcounter{tocdepth}{1} % Uncomment this line if you don't want subsections listed in the ToC

\newpage
\tableofcontents
\newpage

\section{General Setup}

The first decision our team made after we initially heard of the assignment task,was to use Java as our programming language. This decision was made mainly out of familiarity with the language and because we couldn't find any caveats that would complicate fulfilling the assignment.

We also chose to share our code through a Git repository, which you can access through:

\begin{itemize}
\item \url{https://github.com/dom2503/Quarto}
\end{itemize}

We also wanted to start out with a basic command line interface as it was shown during the lecture and create a graphical interface maybe later on, if there would be time left.

You can see an example of the interface we created in \autoref{fig:basic-interface}.

 \begin{figure}[bth]
 \includegraphics[width=1.0\linewidth]{graphics/basic-interface.png}
\caption{Basic interface of our Quarto game.}\label{fig:basic-interface}
 \end{figure}

The general structure of our code is pretty simple. The main class of the application is \textit{QuartoGame}, which contains the game loop and almost all setup and command line interface parts. The actual game is put together through the \textit{Board} and \textit{Piece} classes and some additional enums for the properties of the pieces.

All the different players that we were supposed to write just needed to be able to pick a next piece and to make a move with the piece that was given to them, as can be seen in the shortened version of our \textit{Player} interface below.

\begin{listing}[H]
\caption{Player interface}
\begin{minted}[linenos=true]{java}
public interface Player {
  public String makeMove();
  public Piece selectPieceForOpponent();
  public void setGivenPiece(Piece givenPiece);
}
\end{minted}
\end{listing}

\section{Evaluation Function}

We first started out thinking about the actual evaluation function for the states of the Quarto game, after we finished the main game logic, the random and the human player. As the main focus was to get the Minimax algorithm working, we wrote a very basic dummy version that just returned a random double value in the beginning.

\begin{listing}[H]
\caption{Dummy evaluation function}
\begin{minted}[linenos=true]{java}
public double evaluateBoard(Board board){
        return rand.nextDouble();
}
\end{minted}
\end{listing}

In the real version later on it should evaluate, whether the current state of the board is good for the last player that made a move though. A positive value meaning that this is the case and a negative one meaning that the opponent has an advantage.

The state of the game as we look at it, is just the board and the pieces that were already placed.

After our implementation had developed into a more mature state we added the two distinctive finishing states of the Quarto game: a draw and the win of the player that set the last piece. As you can see below, we were returning positive infinity for a win, because it is the best outcome that can be reached and 0.0 for a draw, because no player has an advantage there.

\begin{listing}[H]
\caption{Very basic evaluation function}
\begin{minted}[linenos=true]{java}
public double evaluateBoard(Board board){
    double result;
    if(board.gameWasWon()){
      result = Double.POSITIVE_INFINITY;
    } else if(board.isDraw()){
      result = 0.0;
    } else {
      result = rand.nextDouble();
    }

    return result;
}
\end{minted}
\end{listing}

Figuring out the next evolutionary step of the evaluation process proofed to be a bit more difficult though. What we somehow concluded was, that leaving a lot of nearly completed rows for the next player is somehow a bad decision.

\section{Novice vs. Random}

\section{Novice vs. Minimax-3}

\section{Minimax-3 vs. Minimax-4}

\section{Tournament experiences}

\end{document}
